\documentclass[11pt]{article}
\usepackage{fullpage}
\usepackage{multirow}
% \usepackage{color}
% \usepackage{graphicx}
% \usepackage{epsfig}


\usepackage{url}
\usepackage{natbib}
\usepackage{hyperref}
\newcommand{\mailto}[1]{\href{mailto:#1}{#1}}

\setcitestyle{comma,numbers,sort&compress,square}

\title{Semantic Delta-Debugging\\
Midpoint Review}
\author{Andrew Coonce, Suhail Shergill, Tristan Ravitch \\
\{\mailto{coonce}, \mailto{shergill}, \mailto{travitch}\}@cs.wisc.edu
}
\begin{document}
% \special{papersize=8.5in,11in}
% \setlength{\pdfpageheight}{\paperheight}
% \setlength{\pdfpagewidth}{\paperwidth}
% You may need to change the horizontal offset to do what you
% want.  Setting \hoffset to a negative value moves all printed
% material to the left on all pages; setting it to a positive value
% moves all printed material to the right on all pages; not setting
% it keeps all printed material in it's default position.  \voffset
% is the vertical offset: use negative value for up; don't set if
% you want to use default position; use positive for down.
% \hoffset = -0.2truein
% \voffset = -0.2truein
\maketitle

\section{Progress to Date}
As set forth in our initial proposal, we continue to view the dependency
detection algorithm as decoupled from the delta-debugging
algorithm. Accordingly, our work to date has emphasized the dependency detection
aspect of the project and this midpoint review will detail our progress on that
section.

The \emph{front-end} (responsible for parsing source files and generating
semantic dependencies between source ranges) is the source of most of the
progress that we have made. Given the need to intelligently walk the AST, we
have extended the C/C++ front-end of Clang to introduce our syntactic and
semantic dependency constrains over \emph{source ranges}. As anticipated, we
have encountered some character-level source range issues in Clang that required
additional character-level parsing, but by-and-large the existing AST has
provided the majority of the required information.

We have encountered a few minor issues during development, but these were
anticipated in the initial project proposal. As there was a proof-of-concept
program sketch made last summer, we knew that the source ranges reported by
Clang \citep{clang} sometimes did not encompass what one would expect. For
example, the source range of a \emph{typedef} declaration statement (that is, a
TypedefDecl node) does not include the initial "typedef" keyword nor does it
extend to the trailing semicolon. To address this, we implemented a few helper
functions to generate "true" source ranges for these few special cases. As we
had anticipated encountering these kinds of issues, when they arose we were able
to handle them in time and have remained on schedule.

Our output from the Constraint Generation phase is a flat-file representing the
mapping from symbols (which can represent statements, variables, types, etc.) to
source-ranges and the dependencies between symbols. All this is in the form of
Prolog predicates.


Thus far, we have developed the following components of our front-end:

\begin{itemize}
\item{\emph{Driver} - initializes the diagnostic components and envokes the
  compiler}
%% i'm not sure 'envokes the compiler' is the right phrase we want here. --- SSS

%% \item{\emph{DeclForTypeVisitor} - visits Decl objects (the Clang representations
%%   of top-level function/variable/etc. declarations) }
%% don't need to make special mention of this. All it did was return the
%%   declaration object corresponding to a Type.

\item{\emph{ConstraintVisitor} - visits Stmt objects (the Clang representation
  of statements/expressions) and creates dependencies between a statement and
  the declaration of the variables referenced therein.}

%% might want to give this (ConstraintVisitor) a new name? one which is more
%% informative --- SSS

\item{\emph{ConstraintGenerator} - visits the top-level variable, function and
  type declarations (typedefs) and generates appropriate dependencies. Also
  is responsible for instantiating a fresh statement visitor to visit the body
  of each of the top level functions.}

%% \item{\emph{GenerateConstraintsAction} - I HAVE NO IDEA WHAT THIS IS... PLEASE
%%   ADVISE - ALC}
%% It's just an argument parser for the driver along with an initializer. We
%%   don't need to mention it. --- SSS

\item{\emph{Helper functions} - Clang doesn't quite give us the sourceranges we
  want for the various lexical elements and so we have helper functions in place
  to scan forward/backward in the source file. There are also helper methods in
  place responsible for printing out the Prolog predicates encapsulating source
  range information and dependencies between various elements.
\end{itemize}

The visitors, which represent the majority of the complexity at this stage of
the project, are largely complete. We have a full-featured declaration-level
dependency generator that handles \emph{typedef}, \emph{enum}, \emph{struct},
\emph{global variable}, and \emph{function} declaration related
dependencies. Within the function declaration dependency generator, our
statement-level visitor traverses the statements and expressions within the
function scope. This statement-level visitor currently handles \emph{variable}
dependencies and appropriately resolves the \emph{scope} of dependencies as
well.

We've also started tackling some of the more important C++ constructs and have
dependency-generating visitors in place for \emph{function templates} and
\emph{template specialization}. We're currently working on handling
\emph{classes} generating dependencies due to \emph{friendship} and
\emph{inheritance}. While not complete for all language features to date, we
anticipate finishing the Constraint Generation phase of the project for most of
the important constructs by our 11/12 deadline.


\section{Future Developments}
%alc - Suhail, I could use your opinion on this part

Our next project phase, the Querying Framework, will involve the design and
implementation of the search and querying strategy. Encapsulating the dependency
information in the form of Prolog predicates means that things like computing
the transitive closure of the dependency relation will be trivial. Over the
course of the next few weeks we'll consider possible search strategies and how
to go about implementing them. We will also consider if there would be any
benefit in going to a graph based representation whereby we might be able to
make use algorithms from graph cutting. For this next phase of the project we
plan on working with PySWIP \citep{pyswip} which provides a bridge between
Python and SWI-Prolog.


%% Working from the dependency
%% graph, we will develop a graph partitioning strategy that approximates the
%% binary search of a traditional delta debugging implementation while drawing upon
%% the wealth of dependecy-related information provided by the Constraint
%% Generation phase. At this stage, we are considering extending our framework to
%% handle more C++ language features as changes to handling of specific AST node
%% "types" is a relatively painless process given the general nature of our output
%% format and the general robustness of the delta debugging algorithm.

%alc - should we mention that we are pulling together test cases for our evaluation phase?
%alc - should we mention our evaluation phase?

%% yeah put in a line or two about us looking at test cases and using those as a
%% guide to see which language features we want to prioritize for our constraint
%% generation step. --- SSS

\bibliographystyle{abbrvnat}
\bibliography{midpointRefs}

\end{document}
