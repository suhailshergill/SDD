\documentclass[11pt]{article}
\usepackage{fullpage}
\usepackage{multirow}
% \usepackage{color}
% \usepackage{graphicx}
% \usepackage{epsfig}


\usepackage{url}
\usepackage{natbib}
\usepackage{hyperref}
\newcommand{\mailto}[1]{\href{mailto:#1}{#1}}

\setcitestyle{comma,numbers,sort&compress,square}

\title{Semantic Delta-Debugging\\
Midpoint Review}
\author{Andrew Coonce, Suhail Shergill, Tristan Ravitch \\
\{\mailto{coonce}, \mailto{shergill}, \mailto{travitch}\}@cs.wisc.edu
}
\begin{document}
% \special{papersize=8.5in,11in}
% \setlength{\pdfpageheight}{\paperheight}
% \setlength{\pdfpagewidth}{\paperwidth}
% You may need to change the horizontal offset to do what you
% want.  Setting \hoffset to a negative value moves all printed
% material to the left on all pages; setting it to a positive value
% moves all printed material to the right on all pages; not setting
% it keeps all printed material in it's default position.  \voffset
% is the vertical offset: use negative value for up; don't set if
% you want to use default position; use positive for down.
% \hoffset = -0.2truein
% \voffset = -0.2truein
\maketitle

\section{Progress to Date}
As set forth in our initial proposal, we continue to view the dependency
detection algorithm as decoupled from the delta-debugging
algorithm. Accordingly, our work to date has emphasized the dependency detection
aspect of the project and this midpoint review will detail our progress on that
section.

The \emph{front-end} (responsible for parsing source files and generating
semantic dependencies between source ranges) is the source of most of the
progress that we have made. Given the need to intelligently walk the AST, we
used Clang \citep{clang} libraries to parse C/C++ code and introduce our
syntactic and semantic dependency constrains over \emph{source ranges}. As
anticipated, we have encountered some minor source range issues in Clang that
required additional character-level parsing, but by-and-large the AST methods as
provided by Clang have provided the majority of the required information.

As there was a proof-of-concept program sketch made last summer, we knew that
the source ranges reported by Clang sometimes do not encompass what one would
expect. For example, the source range of a \emph{typedef} declaration statement
(that is, a TypedefDecl node) does not include the initial ``typedef'' keyword,
nor does it extend to the trailing semicolon. To address this, we implemented a
few helper functions to generate ``true'' source ranges for these few special
cases. As we had anticipated encountering these kinds of issues, when they arose
we were able to handle them in time and have remained on schedule.

Our output from the Constraint Generation phase is a flat-file representing the
both the mapping from symbols (statements/functions/etc.) to source-ranges and
the dependencies between symbols. We record this information in the form of
Prolog predicates that will feed directly into the next project phase.


Thus far, we have developed the following components of our front-end:

\begin{itemize}
\item{\emph{Driver} - initializes and envokes the front-end.}

\item{\emph{StmtVisitor} - visits Stmt objects (the Clang representation
  of statements/expressions) and creates dependencies between a statement and
  the declaration of the variables referenced therein.}

\item{\emph{ConstraintGenerator} - visits the top-level variable, function and
  type declarations (typedefs) and generates appropriate dependencies. Also
  is responsible for instantiating a fresh statement visitor to visit the body
  of each of the top level functions.}

\item{\emph{Helper functions} - Clang doesn't quite give us the source ranges we
  want for the various lexical elements so we have added helper functions
  to scan forward/backward in the source file. There are also helper methods in
  place responsible for printing out the Prolog predicates encapsulating source
  range information and dependencies between various elements.
\end{itemize}

The visitors, which represent the majority of the complexity at this stage of
the project, are largely complete. We have a full-featured declaration-level
dependency generator that handles \emph{typedef}, \emph{enum}, \emph{struct},
\emph{global variable}, and \emph{function} declaration related
dependencies. Within the function declaration dependency generator, our
statement-level visitor traverses the statements and expressions within the
function scope. This statement-level visitor currently handles \emph{variable}
dependencies and appropriately resolves the \emph{scope} of dependencies as
well.

We've also started tackling some of the more important C++ constructs and have
dependency-generating visitors in place for \emph{function templates} and
\emph{template specialization}. We're currently working on handling
\emph{friendship} and \emph{inheritance} related \emph{class}
dependencies. While not complete for all language features to date, we
anticipate finishing the Constraint Generation phase of the project for most of
the important constructs by our 11/12 deadline.


\section{Future Developments}

Our next project phase, the Querying Framework, will involve the design and
implementation of the search and querying strategy. Encapsulating the dependency
information in the form of Prolog predicates means that things like computing
the transitive closure of the dependency relation will be trivial. Over the
course of the next few weeks, we will consider possible search strategies and
how they might be implemented. We will also consider whether there would be any
benefit in going to a graph based representation whereby we might be able to
make use algorithms from graph cutting. For this next phase of the project we
plan on working with PySWIP \citep{pyswip} which provides a bridge between
Python and SWI-Prolog. Finally, we have begun looking at specific test cases for
use in the evaluation phase with the intent of incorporating as many of the language
features as we can to allow us to run our algorithm on real-world code.

\bibliographystyle{abbrvnat}
\bibliography{midpointRefs}

\end{document}
